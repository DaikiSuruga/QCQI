\begin{flushleft}
\section{\Large Exercise 2.1:(Linear dependence: example) }
Show that (1, -1), (1, 2) and (2, 1) are linearly dependent.
\vspace{0.1in}

{\large proof:}
\begin{equation}
a_1\left(
	\begin{array}{c}
	1\\
	-1
	\end{array}
\right)
+ a_2\left(
	\begin{array}{c}
	1\\
	2
	\end{array}
\right)
+ a_3
\left(
	\begin{array}{c}
	2\\
	1
	\end{array}
\right)
= 0
, (a_1, a_2, a_3 \in \mathbb{C})
\end{equation}
とする。
\newline
この解の組$(a_1, a_2, a_3)$は$t\in \mathbb{C}$を任意の複素数として、
\begin{eqnarray}
(a_1, a_2, a_3) = (t, t, -t)
\end{eqnarray}
と表すことが出来る。問題の三つのベクトルが一次独立であれば、
$(a_1, a_2, a_3) = (0, 0, 0)$のみが解となるため不適。
よって一次従属である。
\end{flushleft}

