\begin{flushleft}
{\Large Exercise 2.2: (Matrix represetations: example)}
 Suppose $V$ is a vector space with basis vectors $\ket{0}$ and $\ket{1}$,
and $A$ is a linear operator from $V$ to $V$ such that 
$A\ket{0} = \ket{1} ~and~ A\ket{1} = \ket{0}.$ 
Give a matrix representation for $A$, with respect to the input basis 
$\ket{0}, \ket{1},$ and the output basis $\ket{0}, \ket{1}.$ 
Find input and output bases which give rise to a different matrix representation of $A$.

\vspace{0.1in}
{\large proof:}
 $V$の基底
\{$
 \ket{v_1} = \ket{0}, \ket{v_2}=\ket{1} 
$
\}に対して演算子$A:V \to V $は、
\setcounter{equation}{0}
\begin{eqnarray*}
A\ket{v_1} &=& 0\ket{v_1} + 1\ket{v_2},\\
A\ket{v_2} &=& 1\ket{v_1} + 0\ket{v_2}
\end{eqnarray*}
という作用をする。
\newline
定義より、演算子$A:V \to W $において、
$V$の基底を$\{\ket{v_1},\cdots, \ket{v_n}\}$, 
\newline
$W$の基底を$\{\ket{w_1},\cdots, \ket{w_m}\}$
\newline
とすれば、$A$の行列表現における(i, j)成分$A_{ij}$は
\begin{eqnarray}
A\ket{v_j} = \sum_{i} A_{ij} \ket{w_i}
\end{eqnarray}
と定義されている。
 故に
\begin{eqnarray*}
A_{11}=0,
A_{12}=1,
A_{21}=1,
A_{22}=0.
\end{eqnarray*}
\newline
すなわち、
\[
A = \left(
	\begin{array}{cc}
	0 & 1\\
	1 & 0
	\end{array}
	\right).
\]

 \end{flushleft}

