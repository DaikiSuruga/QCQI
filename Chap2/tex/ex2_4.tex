\setcounter{equation}{0}
\begin{flushleft}
\section{\Large Exercise 2.4: (Matrix representation for identity)}
Show that the identity operator on a vector space $V$ has a matrix representation 
which is one along the diagonal and zero everywhere else, if the matrix representation
is taken with respect to the same input and output bases.
This matrix is known as the identity matrix.
\vspace{0.1in}
\newline
{\large proof:}
ベクトル空間$V$の基底を$\{\ket{v_1}, \cdots, \ket{v_n}\}$, 
$V$上のidentity operatorを$I$とする。
このとき、任意の$j, (1\leq j\leq n)$について
\begin{eqnarray*}
I\ket{v_j}  
&=& \sum_{i=0}^n a_{ij} \ket{v_i} \\
&=& \ket{v_j}.
\end{eqnarray*}
すなわち、
 \[a_{ij} = \begin{cases}
	1 ~(i = j)\\
	0 ~(i \neq j)
	\end{cases}
\]
このような数を$(i,j)$成分に持つ行列は、対角成分が1で
その他が0の行列である。
\end{flushleft}

