{\large 復習:(表現行列の定義)}
\setcounter{equation}{0}
線形写像$f:V \to W$において、
$V$の基底を$\{v_1, \cdots, v_n\}$, $W$の基底を$\{w_1,\cdots, w_m\}$とする。
\newline
ここで,任意の$j (1\leq j\leq n)$について、
$f(v_j) = \sum_{i=1}^m a_{ij}w_i$とする。
\newline
($f:V \to W$より、任意のj($1\leq j\leq n$)について
$f(v_j)$は$W$の基底$\{w_1, \cdots, w_m\}$の線型結合で表される。
)
\newline
このとき行列$A$の(i, j)成分を$a_{ij}$とすれば、 
\begin{equation}
[f(v_1), \cdots, f(v_n)] = [w_1, \cdots, w_m]A
\end{equation}
というような便宜的表記をすることができる。この行列$A$を$f$の表現行列という。
\newline
 この表現行列の便利な点
\newline
$V$の基底を$\{\ket{v_1},\cdots, \ket{v_n}\}$とすると、 
$V$の任意の元$\upsilon$は
$\upsilon = \sum_{j=1}^n x_j v_j, (\forall j, x_j \in \mathbb{C})$
として表すことが出来る。
\newline
\begin{equation}
f(\upsilon) \newline
=f(\sum_{j=1}^n x_j v_j) \\ 
=\sum_{j=1}^n x_j f(v_j) \\
=\sum_{j=1}^n x_j \sum_{i=1}^m a_{ij}w_i \\
=\sum_{j=1}^n \sum_{i=1}^m x_j a_{ij}w_i \\
=\sum_{i=1}^m (\sum_{j=1}^n a_{ij} x_j)w_i
\end{equation}
ここで、
\[x = \left(
	\begin{array}{c}
	x_1 \\
	\vdots \\
	x_n
	\end{array}
	\right)
\]
とすると、
$\sum_{j=1}^n a_{ij} x_j$
は(m, 1)ベクトル$Ax$の$i$成分となる。
\newline
ここで特に(2)において、
\begin{equation}
f(\upsilon)
=\sum_{j=1}^n x_j f(v_j) \\
=\sum_{i=1}^m (\sum_{j=1}^n a_{ij} x_j)w_i
\end{equation}
が成立していることに注意すると、
(1)の表記を用いて
\begin{equation}
[f(\upsilon)]=[f(v_1), \cdots, f(v_n)]x
 = [w_1, \cdots, w_m]Ax
\end{equation}
と表すことが出来る。
すなわち、$V$の基底$\{v_1,\cdots, v_n\}$に対する$\upsilon$の係数ベクトル$x$
で $W$の基底$\{w_1,\cdots, w_m\}$の係数を$Ax$と表すことが出来る。
これが表現行列の便利な点である。
