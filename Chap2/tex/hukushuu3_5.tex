\hypertarget{hukushuu3.5}{}
\section{\Large 復習3.5:(エルミート変換の一意性)}
任意の内積空間$(V, (\cdot, \cdot))$に作用する
任意の線形作用素$A$に対して、次を満たすような線形作用素$A^\dagger$が
一意に定まる。
\begin{eqnarray*}
\forall \ket{x}, \ket{y} \in V, ~~~
(\ket{x}, A \ket{y}) = (A^\dagger \ket{x}, \ket{y}).
\end{eqnarray*}
{\large proof:}
まず、$A^\dagger$の存在を示す。
\newline
$V$の正規直交基底を任意に一つ定め、$<e_1, \cdots, e_n>$とする。
ここで、$x \in V$に対して、$A^\dagger x = \sum_{i}^n (Ae_i, x) e_i$
と置けば、$A^\dagger$は$V$の線形変換である。
このとき、任意の$y \in V$に対して、$y = \sum_{i}^n y_i e_i$とすると、
\begin{eqnarray*}
(A^\dagger x, y) = (\sum_{i}^n (Ae_i, x) e_i, y)
= \sum_{i}^n (Ae_i, x)^*  (e_i, y)
= \sum_{i}^n y_{i} (x, Ae_i).
\end{eqnarray*}

また、
\begin{eqnarray*}
(x, Ay) = (x, \sum_{i}^n y_i A e_i) = \sum_{i}^n y_i (x, Ae_i)
\end{eqnarray*}
ゆえに、任意の$x, y \in V$について
$(A^\dagger x, y) = (x, Ay)$が成り立つ。
\newline
ここからは一意性を示す。
$f, g$を$V$から$V$への写像であるとする。
任意の$x, y \in V$に対して
$(f(x), y) = (g(x), y)$
が成り立つとすると、
$0 = (f(x), y) - (g(x), y) = (f(x) - g(x), y)$
であるが、ここで特に$y = f(x) - g(x)$と置けば
$\|f(x) - g(x) \|^2 = (f(x) - g(x), f(x) - g(x)) = 0$ 
となる。よって$f(x) - g(x) = 0$である。
したがって$f(x) = g(x)$であり、$x$は任意だから$f=g$である。
これより
特に任意の$x, y \in V$に対して、
$(A_{1}^\dagger x, y) = (x, Ay) = (A_{2}^\dagger x, y)$ 
であるから、上記で示したとおり$A_{1}^\dagger = A_{2}^\dagger$が分かる。
