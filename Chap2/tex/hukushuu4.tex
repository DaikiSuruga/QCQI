\hypertarget{hukushuu4}{}
\section{\Large 復習4(正規行列の性質):}
複素正方行列$A$が正規行列であるとは,$A^\dagger A = AA^\dagger$
を満たすことを言う。
\newline
\begin{itemize}
\item {\large 性質1:}$A$を正規行列とし、$A$の固有値$\alpha \in \mathbb{C}$
に対する固有空間を$V_{\alpha}$とするとき、$x \in V_{\alpha}^\perp$
ならば$Ax \in V_{\alpha}^\perp$が成立する。
\newline
{\large proof:}
$x \in V_{\alpha}^\perp, y \in V_{\alpha}$とすると、
$A(A^\dagger y) = A^\dagger (Ay) = A^\dagger (\alpha y) = \alpha (A^\dagger y)$
なので、$A^\dagger y \in V_{\alpha}$である。
よって、$(Ax) \cdot y = x \cdot ( A^\dagger y) = 0.$
\newline
\item {\large 性質2:}$A$を正規行列,$P$がユニタリ行列とすると、
$B = P^{-1}AP = P^\dagger A P$
は正規行列である。
\newline
{\large proof:}
\begin{eqnarray*}
BB^\dagger &=& (P^\dagger AP)(P^\dagger AP)^\dagger\\
 &=& (P^\dagger AP)(P^\dagger A^\dagger P)\\
 &=& P^\dagger AA^\dagger P\\
 &=& P^\dagger A^\dagger AP\\
 &=& P^\dagger A^\dagger PP^\dagger AP\\
&=& (P^\dagger AP)^\dagger (P^\dagger AP)\\
&=& B^\dagger B.
\end{eqnarray*}
ゆえに$B$は正規行列である。
\item {\large 性質3:}複素$n$次正方行列$A$がユニタリ対角化可能$\Leftrightarrow$
$A$が正規行列.
\newline
{\large proof:}
まず、次の補題を証明しておく。
\newline
\color{black}
補題:
任意のベクトル空間$V$と
その部分空間$W$について、$V = W \oplus W^\perp$.
ここで、$\oplus$ は直和を表す。
\newline
\color{blue}
(表記について:
\begin{itemize}
\item $Vがその部分空間W_1, \cdots, W_kの直和である
\Leftrightarrow
\newline
V = \{w_1 + w_2 + \cdots w_k | w_i \in W_i, (1 \leq i\leq k)\}$
\newline
かつ
\newline
$w_1 + w_2 + \cdots w_k = 0, ~( w_i \in W_i, (1 \leq i\leq k))$
ならば$w_i = 0, (1 \leq i\leq k)$.
\item
$V$の部分空間$W$の直交補空間$W^\perp$は下式で表される.
\newline
$W^\perp = \{v \in V | w \cdot v = 0, \forall w \in W\}$
)
\end{itemize}
\color{black}
補題の証明:
\newline
$W$の正規直交基底$w_1, \cdots, w_k$を延長した$V$の基底を
$w_1, \cdots, w_k, w_{1}', \cdots, w_{n-k}'$とする。
\newline
\color{blue}
(補足:[基底の延長]
\newline
基底$w_i, (1 \leq i \leq k)$に含まれない$V$のベクトル$w_1'$をとると、
\begin{eqnarray*}
(\sum_i a_i w_i) + a_1'w_1' = 0
 &\Leftrightarrow& (\sum_i a_i w_i) = - a_1'w_1'\\
&\Leftrightarrow& \forall i \in {1, \cdots, k}, a_i = 0, a_1' = 0.\\
&\Leftrightarrow& w_1, \cdots, w_k, w_1'
は一次独立。
\end{eqnarray*}
このように基底を延長することで$V$の基底を作ることができる。
)
\color{black}
\newline
gram-schmidtの直交化により、最初のk個を変えずに正規直交基底を作ることができる。
$V$のベクトル$v = \sum_i \lambda_i w_i + \sum_j \lambda_j' w_j'$
が$W^\perp$に属する条件は、各$w_i, (1 \leq i \leq k)$との内積が0になることである。
\newline
\color{blue}
(補足:
\begin{eqnarray*}
v \in W^\perp 
&\Leftrightarrow& \forall w \in W, v \cdot w = 0\\
&\Leftrightarrow& \forall w \in W, \forall \lambda_i, \lambda_j' \in \mathbb{C},
(\sum_i \lambda_i w_i + \sum_j \lambda_j' w_j') \cdot w = 0\\
&\Leftrightarrow& \forall w \in W, \forall \lambda_i, \lambda_j' \in \mathbb{C},
\sum_i \lambda_i^*  w_i \cdot w + \sum_j \lambda_j^{'*}  w_j' \cdot w = 0\\
&\Leftrightarrow& \forall w \in W, \forall \lambda_i \in \mathbb{C}, 
\sum_i \lambda_i^* w_i \cdot w = 0 ~~(\because 各w_i, w_j'は正規直交基底をなす)\\
&\Leftrightarrow& \forall w \in W, \forall i \in \{1, \cdots, k\},  \lambda_i = 0
\end{eqnarray*}
)
\newline
\color{black}
これは$\lambda_i = 0, (1\leq i\leq k)$と同値。
よって、$W^\perp$の任意のベクトルは$w_j', (1 \leq j\leq n - k)$の線形和で書けるため、
$W^\perp$の正規直交基底は$w_1', \cdots, w_{n - k}'$である。
補題の証明は以上。
\color{red}
\newline
帰納法で証明する。
\newline
$\cdot$ n = 1の時は,$\Leftarrow, \Rightarrow$共に成立している。
\newline
$\cdot$ n $>$ 1とする。まず$\Leftarrow$を証明する。
\newline
$A$を正規行列とする。 
線形変換$f:\mathbb{C}^n \rightarrow \mathbb{C}^n$を
$f(x) = Ax, x \in \mathbb{C}^n$と定義する。
$A$の固有値$\alpha$に対する固有空間$V_{\alpha}$の直交補空間$V_{\alpha}^\perp$
を考える。$V_{\alpha}$の正規直交基底を$\Lambda$, $V_{\alpha}$の次元$dimV_{\alpha} = m$とし、
$V_{\alpha}^\perp$の正規直交基底を$\Upsilon$とすると、$\Lambda, \Upsilon$
を並べて得られる$\Omega$は$\mathbb{C}$の正規直交基底である($\because$補題).
ここで性質1より、$f$の$\Omega$に関する表現行列は、
$U^{-1}AU = \alpha E_m \oplus A_1$で表される。ここで$U$は、
正規直交基底$\Omega$を並べてできるユニタリ行列である。
性質2より、$U^{-1}AU$は正規行列である。これより(n - m)次正方行列$A_1$は正規行列である。
帰納法の仮定から(n - m)次ユニタリ行列$U_1$が存在して、$U_1^{-1}A_1U_1$は対角行列となる。
$P = U(E_m \oplus U_1)$とおくと、
\begin{eqnarray*}
PP^\dagger &=& U(E_m \oplus U_1) (U(E_m \oplus U_1))^\dagger\\
&=& U(E_m \oplus U_1 ) (E_m \oplus U_1)^\dagger U^\dagger\\
&=& U(E_m \oplus U_1 U_1^\dagger) U^\dagger\\
&=& I,\\
P^{-1}AP &=& P^\dagger A P\\
&=& (U(E_m \oplus U_1))^\dagger A (U(E_m \oplus U_1))\\
&=& (E_m \oplus U_1^\dagger) U^\dagger A U (E_m \oplus U_1)\\
&=& (E_m \oplus U_1^\dagger)(\alpha E_m \oplus A_1) (E_m \oplus U_1)\\
&=& \alpha E_m \oplus U_1^\dagger A_1 U_1.
\end{eqnarray*}
ゆえに$P$はユニタリ行列であり、$U_1^\dagger A_1 U_1$が対角行列となることから
$P^{-1}AP = P^\dagger A P$は対角行列となる。
以上より,$\Leftarrow$が示せた。
\newline
$A$について、あるユニタリ行列$U$が存在して、$D = U^{-1}AU$が対角行列であるとする。
この時、性質2より$A = U D U^{-1}$は正規行列となる。 これより$\Rightarrow$が示せた。
\end{itemize}
