\section{\Large 復習5:スペクトル分解(本文p.72)}
$theorem :(Spectral~decomposition)$
Any normal operator $M$ on a vector space $V$ is
diagonal with respect to some orthonormal basis for $V$.
Conversely, any diagonalizable operator is normal.
\newline
{\large proof:}
"正規作用素$M$は対角化可能である"ことを
$V$の次元$d$における帰納法を用いて示す。
\newline
$\cdot ~d = 1$の時、
$V$の基底を$\ket{v}$とする。この時、任意の線形演算子$M$は、基底に対する作用の和
で表されるので、ある$\alpha \in \mathbb{C}$を用いて
$M\ket{v} = \alpha \ket{v}$と表すことができる。ゆえに、
一次元に作用する線形演算子$M$はその基底$\ket{v}$で対角化可能で、
$M = \alpha \ket{v}\bra{v}$と表すことができる。
\newline
~$\lambda$を$M$の固有値とし、$P_\lambda$を対応する固有空間の正規化された
固有ベクトル$\ket{\lambda}$への射影作用素とし、
$Q_\lambda$をその補空間への射影作用素とする。
つまり、$P = \ket{\lambda} \bra{\lambda}, Q_\lambda = I_V - P$
である。
この時、$M = IMI = 
(P_\lambda  + Q_\lambda) M (P_\lambda + Q_\lambda) 
= P_\lambda MP_\lambda + Q_\lambda M P_\lambda + P_\lambda M Q_\lambda 
+ Q_\lambda M Q_\lambda .$
明らかに,$PMP = \lambda P$. また、$QMP = 0.$
ここで、$\ket{v}$を部分空間$P$のベクトルとすれば、
$MM^\dagger \ket{v} = M^\dagger M \ket{v} = \lambda M^\dagger \ket{v}.$
ゆえに$M^\dagger \ket{v}$は固有値$\lambda$をもつため、部分空間$P$の要素である。
すなわち$QM^\dagger P = 0.$
これのエルミート共役(自己随伴共役)をとれば,$PMQ = 0.$
これらより、$M = PMP + QMQ.$
\newline
次に、$QMQ$が正規であることを示す。
$QM = QM(P + Q) = QMQ,~ QM^\dagger = QM^\dagger (P + Q) = QM^\dagger Q.$
また、射影作用素$Q$の性質$Q^2 = Q$より、
\begin{eqnarray*}
QMQQM^\dagger Q &=& QMQM^\dagger Q~(\because Q^2 = Q)\\
&=& QMM^\dagger Q~(\because QMQ = QM)\\
&=&QM^\dagger M Q~(\because MM^\dagger = M^\dagger M)\\
&=& QM^\dagger QMQ~(\because QM^\dagger = QM^\dagger Q)\\
&=& QM^\dagger QQMQ~(\because Q^2 = Q).
\end{eqnarray*}
ゆえに,$QMQ$は$(d - 1)$次元に作用する正規作用素である。
(詳しく言えば、$QMQ$は、
$\ket{\lambda}$を含む正規直交基底から、
$\ket{\lambda}$を除いた基底で張られる空間に作用する正規作用素である。)
\newline
ここで、$(d - 1)$次元に作用する$QMQ$の固有値$\nu$と対応する固有値$\ket{\nu}$
について考える。
$QMQ\ket{\nu} = QM\ket{\nu} = \nu \ket{\nu}$.
$Q^2 = Q$に注意すれば、
$Q^2 M \ket{\nu} = Q (\nu \ket{\nu}) = \nu \ket{\nu}$.
ゆえに帰納法の仮定より、$QMQ$は、$Q$の正規直交基底を用いて対角化可能である。
また、$PMP = \lambda P$より、$P$の正規直交基底で対角化されている。
これらより、$M = PMP + QMQ$はベクトル空間全体の正規直交基底で対角化可能である。

