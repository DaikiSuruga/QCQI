\section{\Large 復習5:スペクトル分解(本文p.72)}
$theorem :(Spectral~decomposition)$
Any normal operator $M$ on a vector space $V$ is
diagonal with respect to some orthonormal basis for $V$.
Conversely, any diagonalizable operator is normal.
\newline
{\large proof:}
"正規作用素$M:V \rightarrow V$は対角化可能である"ことを
$V$の部分空間$W$の次元$d$における帰納法を用いて示す。
"正規作用素$M$は対角化可能である"ことを
$V$の次元$d$における帰納法を用いて示す。
\newline
$\cdot ~d = 1$の時、
$W$の基底を$\ket{v}$とする。この時、$V$で定義された
任意の$W \subset V$上の正規作用素
$M:V \rightarrow V$の作用は、基底に対する作用の和
で表されるので、ある$\alpha \in \mathbb{C}$を用いて
$M\ket{v} = \alpha \ket{v}$と表すことができる。
ゆえに、一次元に作用する正規作用素$M$は対角化可能で、
$M\ket{v} = \alpha \ket{v}$と表すことができる。ゆえに、
一次元に作用する線形演算子$M$はその基底$\ket{v}$で対角化可能で、
$M = \alpha \ket{v}\bra{v}$と表すことができる。
(ここで大事なのは、$W \setminus V$における$M$の作用は気にしないことである。
とにかく、$W$上において、$M$は$M = \alpha \ket{v}\bra{v}$と
同じ作用をすることを示したのである。)
\newline
$\cdot Vの次元をdとする。$
$W$を次元が$(d - 1)$であるような、$V$の部分空間としたとき、
正規作用素$M:W \rightarrow W$は$W$の正規直交基底で
対角化可能であるとする。
$\lambda$を$M$の固有値とし、$P_\lambda $を対応する固有空間の正規化された固有ベクトル
$\ket{\lambda}$への
射影作用素とし、$Q_\lambda $をその補空間への射影作用素とする。
つまり、$P_\lambda  = \ket{\lambda} \bra{\lambda}, Q_\lambda  = I_V - P_\lambda $である。

この時、$M = I_VMI_V = 
(P_\lambda   + Q_\lambda ) M (P_\lambda  + Q_\lambda ) 
= P_\lambda  MP_\lambda  + Q_\lambda  M P_\lambda  + P_\lambda  M Q_\lambda  
+ Q_\lambda  M Q_\lambda  .$
明らかに,$P_\lambda MP_\lambda  = \lambda P_\lambda $. また、$Q_\lambda MP_\lambda  = 0.$
ここで、$\ket{v}$を部分空間$P_\lambda $のベクトルとすれば、
~$\lambda$を$M$の固有値とし、$P_\lambda$を対応する固有空間の正規化された
固有ベクトル$\ket{\lambda}$への射影作用素とし、
$Q_\lambda$をその補空間への射影作用素とする。
つまり、$P = \ket{\lambda} \bra{\lambda}, Q_\lambda = I_V - P$
である。
この時、$M = IMI = 
(P_\lambda  + Q_\lambda) M (P_\lambda + Q_\lambda) 
= P_\lambda MP_\lambda + Q_\lambda M P_\lambda + P_\lambda M Q_\lambda 
+ Q_\lambda M Q_\lambda .$
明らかに,$PMP = \lambda P$. また、$QMP = 0.$
ここで、$\ket{v}$を部分空間$P$のベクトルとすれば、
$MM^\dagger \ket{v} = M^\dagger M \ket{v} = \lambda M^\dagger \ket{v}.$
ゆえに$M^\dagger \ket{v}$は固有値$\lambda$をもつような部分空間$P_\lambda $の要素である。
すなわち$Q_\lambda M^\dagger P_\lambda  = 0.$
これのエルミート共役(自己随伴共役)をとれば,$P_\lambda MQ_\lambda  = 0.$
これらより、$M = P_\lambda MP_\lambda  + Q_\lambda MQ_\lambda .$
\newline
次に、$Q_\lambda MQ_\lambda $が正規であることを示す。
$Q_\lambda M = Q_\lambda M(P_\lambda  + Q_\lambda ) = Q_\lambda MQ_\lambda ,~ Q_\lambda M^\dagger = Q_\lambda M^\dagger (P_\lambda  + Q_\lambda ) = Q_\lambda M^\dagger Q_\lambda .$
また、射影作用素$Q_\lambda $の性質$Q_\lambda ^2 = Q_\lambda $より、
\begin{eqnarray*}
Q_\lambda MQ_\lambda Q_\lambda M^\dagger Q_\lambda  &=& Q_\lambda MQ_\lambda M^\dagger Q_\lambda ~(\because Q_\lambda ^2 = Q_\lambda )\\
&=& Q_\lambda MM^\dagger Q_\lambda ~(\because Q_\lambda MQ_\lambda  = Q_\lambda M)\\
&=&Q_\lambda M^\dagger M Q_\lambda ~(\because MM^\dagger = M^\dagger M)\\
&=& Q_\lambda M^\dagger Q_\lambda MQ_\lambda ~(\because Q_\lambda M^\dagger = Q_\lambda M^\dagger Q_\lambda )\\
&=& Q_\lambda M^\dagger Q_\lambda Q_\lambda MQ_\lambda ~(\because Q_\lambda ^2 = Q_\lambda ).
\end{eqnarray*}
ゆえに,$Q_\lambda MQ_\lambda $は$(d-1)$次元の部分空間$W$に作用する正規作用素と考えることができる。
帰納法の仮定より、$Q_\lambda MQ_\lambda :V \rightarrow V$は、
$W$において$Q_\lambda $の正規直交基底を用いて対角化可能である。
\newline
$Q_\lambda MQ_\lambda $の$\ket{x} \in V \setminus W$に対する作用を考えると、$V\setminus W$は
すなわち、$\ket{\lambda}$で張られる空間に等しいため、$Q_\lambda  = I_V - P_\lambda $
であることから、$Q_\lambda MQ_\lambda \ket{x} = 0$であることがわかる。
この議論より、$\lambda_i$ を$Q_\lambda $の正規直交基底とすると、$V$上で
$Q_\lambda MQ_\lambda  = \sum_i \lambda_i \ket{\lambda_i} \bra{\lambda_i}$と表される。
\newline
これらより、$M = P_\lambda MP_\lambda  + Q_\lambda MQ_\lambda  = \lambda \ket{\lambda} \bra{\lambda} 
+ \sum_i \lambda_i \ket{\lambda_i} \bra{\lambda_i}$
はベクトル空間全体の正規直交基底で対角化可能である。
\newline
また、$M = P_\lambda MP_\lambda  + Q_\lambda MQ_\lambda $であることから、$Q_\lambda MQ_\lambda $の固有値を$\nu$,
対応する固有ベクトルを$\ket{\nu}$とすれば、
\begin{eqnarray*}
M\ket{\nu} &=&
P_\lambda MP_\lambda \ket{\nu} + Q_\lambda MQ_\lambda \ket{\nu}\\
&=& Q_\lambda MQ_\lambda \ket{\nu}\\
&=& \nu \ket{\nu}.\\
\end{eqnarray*}
すなわち、$Q_\lambda MQ_\lambda $の固有値、固有ベクトルは$M$の固有値、固有ベクトルであることが分かる。
\newline
これを踏まえずとも、
$M = \sum_i \lambda_i \ket{\lambda_i} \bra{\lambda_i}$
と表されている$M$は、固有値$\lambda_i$とこの値に対応する固有ベクトル
$\ket{\lambda_i}$を持つことが分かる。
ゆえに,$QMQ$は$(d - 1)$次元に作用する正規作用素である。
(詳しく言えば、$QMQ$は、
$\ket{\lambda}$を含む正規直交基底から、
$\ket{\lambda}$を除いた基底で張られる空間に作用する正規作用素である。)
\newline
ここで、$(d - 1)$次元に作用する$QMQ$の固有値$\nu$と対応する固有値$\ket{\nu}$
について考える。
$QMQ\ket{\nu} = QM\ket{\nu} = \nu \ket{\nu}$.
$Q^2 = Q$に注意すれば、
$Q^2 M \ket{\nu} = Q (\nu \ket{\nu}) = \nu \ket{\nu}$.
ゆえに帰納法の仮定より、$QMQ$は、$Q$の正規直交基底を用いて対角化可能である。
また、$PMP = \lambda P$より、$P$の正規直交基底で対角化されている。
これらより、$M = PMP + QMQ$はベクトル空間全体の正規直交基底で対角化可能である。
