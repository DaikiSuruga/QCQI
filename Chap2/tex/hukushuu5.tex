\section{\Large 復習5:スペクトル分解(本文p.72)}
$theorem :(Spectral~decomposition)$
Any normal operator $M$ on a vector space $V$ is
diagonal with respect to some orthonormal basis for $V$.
Conversely, any diagonalizable operator is normal.
\newline
{\large proof:}
"正規作用素$M:V \rightarrow V$は対角化可能である"ことを
$V$の部分空間$W$の次元$d$における帰納法を用いて示す。
\newline
$\cdot ~d = 1$の時、
$W$の基底を$\ket{v}$とする。この時、任意の$V$上の正規作用素$M:V \leftarrow V$
の作用は、
基底に対する作用の和
で表されるので、ある$\alpha \in \mathbb{C}$を用いて
$M\ket{v} = \alpha \ket{v}$と表すことができる。
ゆえに、一次元に作用する正規作用素$M$は対角化可能で、
$M = \alpha \ket{v}\bra{v}$と表すことができる。
\newline
$\cdot Vの次元をdとする。$
$W$を次元が$(d - 1)$であるような、$V$の部分空間としたとき、
正規作用素$M:V \rightarrow V$は,$W$においては$W$の正規直交基底で
対角化可能であるとする。
$\lambda$を$M$の固有値とし、$P_\lambda$を対応する固有空間の正規化された固有ベクトル
$\ket{\lambda}$への
射影作用素とし、$Q_\lambda$をその補空間への射影作用素とする。
つまり、$P_\lambda = \ket{\lambda} \bra{\lambda}, Q_\lambda = I_V - P_\lambda$である。

この時、$M = IMI = 
(P_\lambda  + Q_\lambda) M (P_\lambda + Q_\lambda) 
= P_\lambda MP_\lambda + Q_\lambda M P_\lambda + P_\lambda M Q_\lambda 
+ Q_\lambda M Q_\lambda .$
明らかに,$PMP = \lambda P$. また、$QMP = 0.$
ここで、$\ket{v}$を部分空間$P$のベクトルとすれば、
$MM^\dagger \ket{v} = M^\dagger M \ket{v} = \lambda M^\dagger \ket{v}.$
ゆえに$M^\dagger \ket{v}$は固有値$\lambda$をもつため、部分空間$P$の要素である。
すなわち$QM^\dagger P = 0.$
これのエルミート共役(自己随伴共役)をとれば,$PMQ = 0.$
これらより、$M = PMP + QMQ.$
\newline
次に、$QMQ$が正規であることを示す。
$QM = QM(P + Q) = QMQ,~ QM^\dagger = QM^\dagger (P + Q) = QM^\dagger Q.$
また、射影作用素$Q$の性質$Q^2 = Q$より、
\begin{eqnarray*}
QMQQM^\dagger Q &=& QMQM^\dagger Q~(\because Q^2 = Q)\\
&=& QMM^\dagger Q~(\because QMQ = QM)\\
&=&QM^\dagger M Q~(\because MM^\dagger = M^\dagger M)\\
&=& QM^\dagger QMQ~(\because QM^\dagger = QM^\dagger Q)\\
&=& QM^\dagger QQMQ~(\because Q^2 = Q).
\end{eqnarray*}
ゆえに,$QMQ$は$(d-1)$次元の部分空間$W$に作用する正規作用素と考えることができる。
帰納法の仮定より、$QMQ:V \rightarrow V$は、
$W$において$Q$の正規直交基底を用いて対角化可能である。
\newline
$QMQ$の$\ket{x} \in V \setminus W$に対する作用を考えると、$V\setminus W$は
すなわち、$\ket{\lambda}$で張られる空間に等しいため、$Q = I_V - P_\lambda$
であることから、$QMQ\ket{x} = 0$であることがわかる。
この議論より、$\lambda_i$ を$Q$の正規直交基底とすると、$V$上で
$QMQ = \sum_i \lambda_i \ket{\lambda_i} \bra{\lambda_i}$と表される。
\newline
これらより、$M = PMP + QMQ = \lambda \ket{\lambda} \bra{\lambda} 
+ \sum_i \lambda_i \ket{\lambda_i} \bra{\lambda_i}$
はベクトル空間全体の正規直交基底で対角化可能である。
\newline
また、$M = PMP + QMQ$であることから、$QMQ$の固有値を$\nu$,
対応する固有ベクトルを$\ket{\nu}$とすれば、
\begin{eqnarray*}
M\ket{\nu} &=&
PMP\ket{\nu} + QMQ\ket{\nu}\\
&=& QMQ\ket{\nu}\\
&=& \nu \ket{\nu}.\\
\end{eqnarray*}
すなわち、$QMQ$の固有値、固有ベクトルは$M$の固有値、固有ベクトルであることが分かる。
\newline
これを踏まえずとも、
$M = \sum_i \lambda_i \ket{\lambda_i} \bra{\lambda_i}$
と表されている$M$は、固有値$\lambda_i$とこの値に対応する固有ベクトル
$\ket{\lambda_i}$を持つことが分かる。
