\color{red}{\large 復習:(表現行列の定義)}

線形写像$f:V \to W$において、
$V$の基底を$\{v_1, \cdots, v_n\}$, $W$の基底を$\{w_1,\cdots, w_m\}$とする。
\newline
ここで任意の$j (1\leq j\leq n)$について、
$f(v_j) = \sum_{i=1}^m a_{ij}w_i$とする。
\newline
($f:V \to W$より、任意のj($1\leq j\leq n$)について
$f(v_j)$は$W$の基底$\{w_1, \cdots, w_m\}$の線型結合で表される。
)
\newline
このとき行列$A$の(i, j)成分を$a_{ij}$とすれば、 
\begin{eqnarray*}
[f(v_1), \cdots, f(v_n)] = [w_1, \cdots, w_m]A
\end{eqnarray*}
というような便宜的表記をすることができる。この行列$A$を$f$の表現行列という。
\newline
 この表現行列の便利な点
\newline
$V$の基底を$\{\ket{v_1},\cdots, \ket{v_n}\}$とすると、 
$V$の任意の元$\upsilon$は
$\upsilon = \sum_{j=1}^n x_j v_j, (\forall j, x_j \in \mathbb{C})$
として表すことが出来る。
\newline
\[
f(\upsilon) =
f(\sum_{j=1}^n x_j v_j) = 
\sum_{j=1}^n x_j f(v_j) =
\sum_{j=1}^n x_j \sum_{i=1}^m a_{ij}w_i = 
\sum_{j=1}^n \sum_{i=1}^m x_j a_{ij}w_i
\]
このとき、
